%% Tikz figure
%% Source : https://olivierlemaire.wordpress.com/2010/12/21/frise-chronologique/ (site initial supprimé par son auteur)
%% https://fr.comp.text.tex.narkive.com/FPmUJXhp/frise-chronologique-timeline

%% commandes définies (et à personnaliser) dans le fichier macros/frise.tex
\pgfdeclarelayer{antiground}
\pgfdeclarelayer{background}
\pgfdeclarelayer{foreground}
\pgfsetlayers{antiground,background,main,foreground}

% new commands

\newcommand{\topLegend}[2]{
  node [text centered,%
  %shade,%
  %top color=\fColor,%
  %draw,%
  semithick,
  rounded corners=2pt,
  above,
  #1] {#2}}

\newcommand{\crisisLegend}[2]{
  node [text centered,%
  %shade,%
  %top color=\fColor,%
  %draw,%
  semithick,
  anchor=north,
  rounded corners=2pt,
  above,
  #1] {#2}}
  
\newcommand{\bottomLegend}[2]{
  node [text centered,%
  %shade,%
  %top color=\fColor,%
  %draw,%
  semithick,
  rounded corners=2pt,
  below,
  #1] {#2}}
  
\newcommand{\topEvent}[3]{\draw[<-,very thick,draw=lg] (#1) coordinate (#2) (#2 |- xH axis) -- ++ (#3)}
\newcommand{\bottomEvent}[3]{\draw[<-,very thick,draw=lg] (#1) coordinate (#2) (#2 |- xL axis) -- ++ (#3)}
\newcommand{\bottomGraphics}[2]{node[draw,anchor=north,inner sep=1pt,thin] {\includegraphics[#1]{#2}}}
\newcommand{\topGraphics}[2]{node[draw,anchor=south,inner sep=1pt,thin] {\includegraphics[#1]{#2}}}

\resizebox{\textwidth}{!}{%
\begin{tikzpicture}[%
=latex,%
font=\fontfamily{phv}\fontsize{7}{8}\selectfont,
legend/.style={%
draw,
rounded corners=2pt,
shade}]

% frise
\draw[rounded corners,shade,left color=blue_dark!40,right color=blue_dark!5] (-14,0) -- ++ (28,0) -- ++ (1,1.5) -- ++(-1,1.5) -- ++ (-28,0) -- cycle;
% axe bas
\path (-10,0) -- (10,0) coordinate (xL axis);
% axe haut
\path[yshift=+3cm] (-10,0) -- (10,0) coordinate (xH axis);
% rupture
\path[] (1.25,-0.25) rectangle ++(5mm,3.5) node [midway,rotate=90,minimum width=3.5cm,shade,sharp corners,inner sep=1pt] (r1) {\tiny rupture};
\draw[densely dotted] (r1.south east) -- (r1.south west) (r1.north east) -- (r1.north west);
\draw[] (0,0) coordinate (an0) -- (an0 |- xH axis) -- ++ (0,2.5mm) node [above] {An 0};


% scope 1
\begin{scope}[xshift=3cm,xscale={14cm/400cm}]
% graduation
\draw[] (0,0) coordinate (an0) -- (an0 |- xH axis) -- ++ (0,2.5mm) node [above] {1600};
\draw[] (100,0) coordinate (an100) -- (an100 |- xH axis) -- ++ (0,2.5mm) node [above] {1700};
\draw[] (200,0) coordinate (an200) -- (an200 |- xH axis) -- ++ (0,2.5mm) node [above] {1800};

% evennement
\bottomEvent{42,0}{an1642}{0,-5mm} \bottomLegend{text width=2cm,name=p}{1642 :\\Pascaline};
\topEvent{17,0}{an1617}{0,1} \topLegend{text width=4cm,name=r}{1617 :\\Napier : invention logarithmes\\W. Oughtred : R\`egles \`a calculer};
\bottomEvent{127,0}{an1727}{0,-5mm} \bottomLegend{text width=2cm}{1727 :\\Carte Perfor\'ee};
\topEvent{233,0}{an1617}{0,1} \topLegend{text width=4cm}{1833 :\\machine \`a babbage\\1ere Programmeuse : A. Lovelace};
\end{scope}

% scope 2
\begin{scope}[xshift=0cm,xscale={0.7cm/30cm}]
% graduation
\draw[] (-100,0) coordinate (anM1000) -- (anM1000 |- xH axis) -- ++ (0,2.5mm) node [above,fill=white] {-1000};
\draw[] (-200,0) coordinate (anM2000) -- (anM2000 |- xH axis) -- ++ (0,2.5mm) node [above,fill=white] {-2000};
\draw[] (-300,0) coordinate (anM3000) -- (anM3000 |- xH axis) -- ++ (0,2.5mm) node [above,fill=white] {-3000};
%evennement
\bottomEvent{-30,0}{anM300}{0,-5mm} \bottomLegend{text width=2.5cm}{-300 :\\Abaques et bouliers};
\topEvent{-30,0}{anM3002}{0,2cm} \topLegend{text width=4cm}{-300 : logique\\syllogismes (\textsc{socrates})\\
\vspace{-12pt}\begin{enumerate}\itemsep-2pt
\item tout homme est mortel
\item or Socrates est un homme
\item donc Socrates est mortel
\end{enumerate}
};
\topEvent{-173,0}{anM1730}{0,7.5mm} \topLegend{text width=2.5cm}{-1730 :\\1er algorithme};
\bottomEvent{-300,0}{anM1730}{0,-5mm} \bottomLegend{text width=3.5cm}{-3000 :\\apparition du binaire dans le symbole de l'empereur chinois Fou~Hi};
\end{scope}
\end{tikzpicture}
}