% Tikz figure
% Source : http://www.texample.net/tikz/examples/gajski-kuhn-y-chart/


\scalebox{0.85}{
\begin{tikzpicture}[>=stealth',join=bevel,font=\sffamily,auto,on grid,decoration={markings,
    mark=at position .5 with \arrow{>}}]
    \coordinate (behaviouralNode) at (135:4cm);
    \coordinate (structuralNode) at (45:4cm);
    \coordinate (physicalNode) at (270:4cm);
    \coordinate (originNode) at (0:0cm);

    \node [above=1em] at (behaviouralNode) {\textbf{Behavioural Domain}};
    \node [above=1em] at (structuralNode) {\textbf{Structural Domain}};
    \node [below=1em] at (physicalNode) {\textbf{Physical Domain}};

    \draw[-, very thick] (behaviouralNode.south) -- (0,0) node[left,pos=0]{Systems} node[left,pos=0.2]{Algorithms} node[left,pos=0.4]{Register transfers} node[left,pos=0.6]{Logic} node[left,pos=0.8]{Transfer functions};

    \draw[-, very thick] (structuralNode.south) -- (0,0) node[pos=0]{ } node[pos=0.2]{Processors} node[pos=0.4]{ALUs, RAM, etc.} node[pos=0.6]{Gates, flip-flops, etc.} node[pos=0.8]{Transistors};

    \draw[-, very thick] (physicalNode.south) -- (0,0) node[right,pos=0]{Physical partitions} node[right,pos=0.2]{Floorplans} node[right,pos=0.4]{Module layout} node[right,pos=0.6]{Cell layout} node[right,pos=0.8]{Transistor layout};

    \draw[fill] (barycentric cs:behaviouralNode=1.0,originNode=0) circle (2pt);
    \draw[fill] (barycentric cs:behaviouralNode=0.8,originNode=0.2) circle (2pt);
    \draw[fill] (barycentric cs:behaviouralNode=0.6,originNode=0.4) circle (2pt);
    \draw[fill] (barycentric cs:behaviouralNode=0.4,originNode=0.6) circle (2pt);
    \draw[fill] (barycentric cs:behaviouralNode=0.2,originNode=0.8) circle (2pt);

    \draw[fill] (barycentric cs:structuralNode=1.0,originNode=0) circle (2pt);
    \draw[fill] (barycentric cs:structuralNode=0.8,originNode=0.2) circle (2pt);
    \draw[fill] (barycentric cs:structuralNode=0.6,originNode=0.4) circle (2pt);
    \draw[fill] (barycentric cs:structuralNode=0.4,originNode=0.6) circle (2pt);
    \draw[fill] (barycentric cs:structuralNode=0.2,originNode=0.8) circle (2pt);

    \draw[fill] (barycentric cs:physicalNode=1.0,originNode=0) circle (2pt);
    \draw[fill] (barycentric cs:physicalNode=0.8,originNode=0.2) circle (2pt);
    \draw[fill] (barycentric cs:physicalNode=0.6,originNode=0.4) circle (2pt);
    \draw[fill] (barycentric cs:physicalNode=0.4,originNode=0.6) circle (2pt);
    \draw[fill] (barycentric cs:physicalNode=0.2,originNode=0.8) circle (2pt);

    \draw[black!50] (0,0) circle (4.0cm);
    \draw[black!50] (0,0) circle (3.2cm);
    \draw[black!50] (0,0) circle (2.4cm);
    \draw[black!50] (0,0) circle (1.6cm);
    \draw[black!50] (0,0) circle (0.8cm);

  \end{tikzpicture}
}