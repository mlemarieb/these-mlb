%%%%%%%%%%%
% Mise en page

% User friendly interface to change layout parameters
% CTAN: http://www.ctan.org/pkg/geometry
\usepackage{geometry}
\geometry{% siehe geometry.pdf (Figure 1)
  top=25mm,
	bottom=25mm,
	showframe=false, % For debugging: try true and see the layout frames
	margin=25mm,
	marginparsep=0mm,
	headsep=1cm,
	footskip = \dimexpr\headsep+\ht\strutbox\relax,
	marginparwidth=15mm%à modifier
}

\reversemarginpar % nécessaire dans le cas où l'option twoside est vraie

%%%%%%%%%%%
% Paragraphes

\usepackage{setspace} % interlignage global (modifié pour les environnements frameenv, tabularx et longtable)
\setstretch{1.125}

% Macros de mise en forme : centrer une page
% debut macro  %
\newenvironment{vcenterpage}
{\vspace*{\fill}\begin{center}}%\newpage
{\end{center}\vspace*{\fill}\par}%\pagebreak
% fin macro % % Macro de nouvelle commande : permettant de centrer un texte en milieu de page (verticalement)
\newenvironment{centerpage}
{\vspace*{\fill}}%\newpage
{\vspace*{\fill}\par}%\pagebreak % Macro de nouvelle commande : permettant de centrer un texte verticalement

\makeatletter%
\@ifclassloaded{scrbook}%
  {%Macro définissant une environnement pour les résumés dans la classe book
% début macro %
\makeatletter
% \newenvironment{myabstract}{%
%     \@beginparpenalty\@lowpenalty
%     \begin{center}%
%       \bfseries \abstractname
%       \@endparpenalty\@M
%     \end{center}}%
%    {\par}
\newenvironment{abstract}%
{\vfill%
{\normalfont\bfseries %\sffamily
{\vspace{0.25cm}\noindent\abstractname}

\vspace{0.25cm}
}%
}%
{\vfill\null}
\makeatother
% fin macro %}%
  {}%
\makeatother%

\RedeclareSectionCommands[
  afterskip=1sp
]{paragraph,subparagraph}

%%%%%%%%%%%
% Autres packages

\usepackage{pdflscape} % Paysage

\usepackage{enumitem} % personnalisation des listes enumerate, itemize (utilisé notamment pour modifier les puces)
% \setlist{nosep,after=\vspace{\baselineskip}} % pour avoir un espace après itemize

\usepackage{epigraph}
\setlength{\epigraphwidth}{0.5\textwidth}
\setlength{\epigraphrule}{0pt}

% %%%%%%%%%%%
% Style des chapitres

\usepackage[usestackEOL]{stackengine}

%\RedeclareSectionCommand[beforeskip=2cm,afterskip=2cm]{chapter}

\makeatletter%
\@ifclassloaded{scrbook}%
  {\definecolor{numbercolor}{gray}{0.7}
\renewcommand*{\chapterformat}{\mbox{\chapappifchapterprefix{\nobreakspace}\scalebox{3}{\selectfont\color{numbercolor}\thechapter}\enskip}}
}%
  {}%
\makeatother%