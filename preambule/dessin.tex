%%%%%%%%%%%%%%%%
%Tikz et dessins

\usepackage{tikz}
\usetikzlibrary{babel} % Fix conflits with babel
\usetikzlibrary{quotes} % Fix conflits with csquotes
\usetikzlibrary{chains} 
\usetikzlibrary{trees} % utile pour dessiner des mindmaps simples
\usetikzlibrary{shapes,shapes.multipart,shapes.geometric,shadows}
\usetikzlibrary{fit,positioning,calc,backgrounds}
\usetikzlibrary{arrows,decorations,decorations.markings}

\usepackage[framemethod=TikZ]{mdframed}
\usepackage{multicol}

% Macros de mise en forme d'encadré : nouvel environnement
% debut macro %
\DeclareFloatingEnvironment[name=Encadré,fileext=loe]{myfloat}% option non utuilisée placement={htbp},
\newenvironment{frameenv}[2]
    {\begin{myfloat}[!p]%htbp
    % \vspace{2ex}
    \begin{mdframed}[roundcorner=5pt,backgroundcolor=colenc!70]%blue!10
    \caption[#1]{#2}
    \begin{multicols}{2}
    \setstretch{1.125} % Interligne
    \footnotesize \glsresetall }
    {\normalsize \end{multicols} \end{mdframed} \end{myfloat}
    }
% fin macro % % macro créant un nouvel environnement flottant de type encadré
% Macros de mise en forme d'encadré continu : nouvel environnement
% debut macro %
%\DeclareFloatingEnvironment[placement={!h},name=Encadré,fileext=loe]{myfloatstar}
\newenvironment{frameenv*}[2]
    {\begin{myfloat}[!p]%htbp
    % \vspace{2ex}
    \begin{mdframed}[roundcorner=5pt,backgroundcolor=colenc!70]%blue!10
    \ContinuedFloat % Suite de l'encadré précédent
    \captionsetup{list=no} % Non mentionné dans la liste des encadrés
    \caption[#1]{#2} % ref,shortcap, longcap
    \begin{multicols}{2}
    \setstretch{1.125} % Interligne
    \footnotesize \glsresetall }
    {\normalsize \end{multicols}\end{mdframed}\end{myfloat}
    }
% fin macro % % macro créant un nouvel environnement flottant de type encadré non numéroté

\setlength{\fboxsep}{1mm}% pour un cadre collé à la fenêtre; on peut même mettre une valeur négative !
%\setlength{\fboxrule}{}% pour régler l'épaisseur du cadre