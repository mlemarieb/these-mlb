
%%%%%%%%%%%%%%%%
  % Bibliographie

\usepackage[style= authoryear-comp, % Style des citations; dans le texte : (Auteur 1, Année, Année ; Auteur 2 Année)
            natbib=true, % compatibilité avec le package natbib
            autolang=other, % différence avec other* ?
            hyperref, % compatibilité avec le package hyperref
            backref, % dans les bibliographies : liens avec les numéros de page où la référence est citée
            backrefstyle=all+,%
            bibencoding=inputenc,
            datamodel=complete-dm,
            abbreviate=false, % abbréviation
            pagetracker=true,
            citecounter=true,
            citereset=subsection,
            url=true, % afficher les url
            isbn=false,doi=false,eprint=false,
            giveninits=true, % Tous les prénoms en initiales % firstinits deprecated
            uniquename=init,  % this option is compatible with firstinits
            uniquelist=false,
            maxcitenames=2,% 2 auteurs listés max lors d'une citation
            maxbibnames=99, %% liste complète des auteurs en biblio
            mergedate=basic, % si la date spécifiée et la date en label sont identiques, les fusionner pour éviter des répétititions: basic
            %datelabel=comp, % 
            date=long,
            urldate=short,
            dateabbrev=false,
            backend=biber % indispensable pour faire fonctionner biber
            ]{biblatex}
\bibliography{ressources/Bibliotheques/biblio_exemple,}

% filtre
% Déclarations des filtres
\defbibfilter{filtrescience}{% Déclaration du filtre documents scientifiques
( type=article or type=incollection or type=inproceedings or type=book  or type=thesis or type=inbook or type=proceedings or type=unpublished or type=report ) and ( not keyword={software} ) and ( not keyword={R-package} )  and ( not keyword={institution report} )  and ( not keyword={statistic report} )} % bien mettre chaque critère entre deux espaces et attention : thesis et non phdthesis 
\defbibfilter{filtrescience_book}{% Déclaration du filtre documents scientifiques
( type=book or type=inbook ) and ( not keyword={software} ) and ( not keyword={R-package} )  and ( not keyword={institution report} )  and ( not keyword={statistic report} )} % bien mettre chaque critère entre deux espaces et attention : thesis et non phdthesis 
\defbibfilter{filtrescience_art}{% Déclaration du filtre documents scientifiques
( type=article or type=thesis or type=inproceedings or type=proceedings ) and ( not keyword={software} ) and ( not keyword={R-package} )  and ( not keyword={institution report} )  and ( not keyword={statistic report} )} % bien mettre chaque critère entre deux espaces et attention : thesis et non phdthesis 
\defbibfilter{filtrerapport}{% Déclaration du filtre rapports institutionnels
( keyword={institution report} ) and ( not keyword={software} ) and ( not keyword={R-package} )}% bien mettre chaque critère entre deux espaces
\defbibfilter{filtreoutils}{% Déclaration du filtre logiciels et packages
( keyword=R-package  or  keyword=software )}% bien mettre chaque critère entre deux espaces
\defbibfilter{filtrelegislatif}{% Déclaration du filtre logiciels et packages
( keyword=law )}% bien mettre chaque critère entre deux espaces

% Déclaration de l'en-tête principal
\makeatletter%
\@ifclassloaded{scrbook}%
  {\defbibheading{ref}{
\chapter*{Références}
\markboth{Références}{Références}}
}%
  {\defbibheading{ref}{
\section*{Références}
\markboth{Références}{Références}}
}%
\makeatother%

% Augmenter l'espace vertical entre les références
\setlength\bibitemsep{0.5ex}
\setlength\bibnamesep{1.2ex}

% Déclaration des nouvelles entrées et nouveaux champs 
\begin{filecontents}{complete-dm.dbx}
%Legislation
\DeclareDatamodelFields[type=field,datatype=literal,skipout=false]{texte}
%Database
\DeclareDatamodelEntrytypes{database}
\DeclareDatamodelFields[type=field,datatype=literal,skipout=false]{alias}
\DeclareDatamodelFields[type=field,datatype=literal,skipout=false]{datacover}
\DeclareDatamodelFields[type=field,datatype=literal,skipout=false]{dataedition}
\DeclareDatamodelEntryfields[database]{author,title,howpublished,url,urldate,datacover,dataedition}
\end{filecontents}

%% Nettoyage des entrées
%% Supprimer les champs inutiles pour tous les types
 \AtEveryBibitem{%
  \clearfield{comment}%
  \clearlist{organization}%
  \clearfield{review}%
  \clearfield{abstract}%
}
%% Supprimer les champs inutiles pour tous les types SAUF pour le type book et incollection
 \AtEveryBibitem{%
  \ifentrytype{book}{}{%
  	\ifentrytype{incollection}{}{%
    		\ifentrytype{report}{}{%
      		\ifentrytype{inproceedings}{}{\clearfield{address}}%
}}}}
%%% Si je veux afficher les url : nettoyage préalable
 \AtEveryBibitem{%
  \ifentrytype{online}{}{%
  \ifentrytype{database}{}{%
    \ifentrytype{legislation}{}{%
  		\ifentrytype{article}{%
 			\iffieldundef{howpublished}{% Si papier
   			\clearfield{month}%
    			\clearfield{note}%
    			\clearfield{day}%
    			\clearfield{url}}{%
    			\clearfield{month}%
    			\clearfield{note}%
    			\clearfield{day}}}{%SI ce n'est pas un article
 		 \ifentrytype{incollection}{%
  			\iffieldundef{howpublished}{% Si papier
    			\clearfield{month}%
    			\clearfield{note}%
    			\clearfield{day}%
    			\clearfield{url}}{%
    			\clearfield{month}%
   			 \clearfield{note}%
   			 \clearfield{day}}}{%SI ce n'est pas un incollection
 		 \ifentrytype{book}{%
  			\iffieldundef{howpublished}{% Si papier
   			\clearfield{month}%
    			\clearfield{note}%
    			\clearfield{day}%
    			\clearfield{url}}{%
    			\clearfield{month}%
    			\clearfield{note}%
    			\clearfield{day}}}{%SI ce n'est pas un book 
 		\ifentrytype{report}{%
  			\iffieldundef{howpublished}{% Si papier
    			\clearfield{month}%
   			\clearfield{note}%
    			\clearfield{day}%
    			\clearfield{url}}{%
    			\clearfield{month}%
    			\clearfield{note}%
    			\clearfield{day}}}{%SI ce n'est pas un report 
  		\ifentrytype{inproceedings}{%
  			\iffieldundef{howpublished}{% Si papier
    			\clearfield{month}%
    			\clearfield{note}%
    			\clearfield{day}%
    			\clearfield{url}}{%
    			\iffieldequalstr{howpublished}{[cd-rom]}{%
    			\clearfield{month}%
    			\clearfield{note}%
    			\clearfield{day}%
    			\clearfield{url}}{}{%Si non cd-rom
    			\clearfield{month}%
    			\clearfield{day}}}%
    			\clearfield{note}}{%SI ce n'est pas un inproceedings 
    		\ifentrytype{newspaper}{%
    			\iffieldequalstr{howpublished}{[en ligne]}{%
    			\clearfield{note}}{%
    			\clearfield{url}%  
    			\clearfield{urldate}}}{% Si rien de tout ça
    \clearfield{note}%
    \clearfield{month}%
    \clearfield{day}%
    \clearfield{url}%  
    \clearfield{howpublished}
    \clearfield{urldate}
  }}}}}}}}}}%

% Paramétrage valable pour tous les typesentrées
% \DeclareLabeldate{%
%   \field{date}
%   \field{eventdate}
%   \field{origdate}
%   \field{urldate}
%   \literal{nodate}
% }
\DeclareNameAlias{sortname}{last-first}% Afficher tous les auteurs en Lastname-Firstname
\DeclareFieldFormat{pagetotal}{\mkpagetotal[bookpagination]{#1~pages}}% Keep abbreviations in general, but use "pages" to format the `pagetotal` field
\DeclareFieldFormat{edition}{\ifinteger{#1}{\mkbibordedition{#1}\addthinspace{}ed.}{#1\isdot}}% réduction des espaces 
% \DeclareFieldFormat{howpublished}{\textbf{#1}}
\DefineBibliographyStrings{french}{%
%  backrefpage = {cité page},
%  backrefpages= {cité pages},
urlseen = {dernier accès le},}
\DefineBibliographyStrings{english}{urlseen = {last access:},}
\DeclareDelimFormat[cbx@textcite]{nameyeardelim}{\addspace}%Fix citet

% Paramétrages spécifiques
\DeclareFieldFormat[article]{number}{(#1)}% number of a journal
\DeclareFieldFormat[legislation]{texte}{texte \no{#1}}
\DeclareFieldFormat[legislation]{volume}{#1}
\DeclareFieldFormat[legislation]{title}{\normalfont #1}
\DeclareFieldFormat[newspaper]{title}{\mkbibquote{#1\isdot}}  % Définir l'affichage du titre d'un article de presse

% Nouvelles commandes de citations
\DeclareCiteCommand{\citetalias}% Citer des alias : BD
  {\booltrue{citetracker}%
   \boolfalse{pagetracker}%
   \usebibmacro{prenote}}
  {\ifciteindex
     {\indexfield{indextitle}}
     {}%
   \printtext[bibhyperref]{\printfield[alias]{alias}}}
  {\multicitedelim}
  {\usebibmacro{postnote}}
\DeclareCiteCommand{\citepalias}[\mkbibparens]% Citer des alias : (BD)
  {\usebibmacro{prenote}}
  {\usebibmacro{citeindex}%
   \printtext[bibhyperref]{\printfield[alias]{alias}}}
  {\multicitedelim}
  {\usebibmacro{postnote}}

%%%%%%%%
% BIBSTYLE: réécriture des commandes

% Uniformisation des formats des noms d'auteur pour toutes les langues
\renewcommand\mkbibnamefamily[1]{\textsc{#1}} % pour que tous les noms d'auteurs (y compris anglo-saxons) soient en petites capitales (convention française)
%\DefineBibliographyExtras{french}{\restorecommand\mkbibnamelast} % pour que tous les noms d'auteurs français en minuscules (convention anglosaxone)

% Pas de point à la fin d'une entrée
\renewbibmacro*{finentry}{\iflistundef{pageref}{\renewcommand{\finentrypunct}{}}{\renewcommand{\finentrypunct}{}}\finentry} 
 
% Supprimer le In: introduit par le style authoryear-comp pour l'entrée Journal Article et newspaper (solution non optimale)
\renewbibmacro{in:}{% 
\ifentrytype{article}{}{%
\ifentrytype{newspaper}{}{%
\ifentrytype{legislation}{}{%
\iffieldundef{booktitle}{}{%
\printtext{\bibstring{in}\intitlepunct}}}}}}

% Personnalisation des drivers : s'appuie sur le fichier : /usr/local/texlive/2014/texmf-dist/tex/latex/biblatex/bbx/standard.bbx
\DeclareBibliographyDriver{article}{%
  \usebibmacro{bibindex}%
  \usebibmacro{begentry}%
  \usebibmacro{author/translator+others}%
  \setunit{\labelnamepunct}\newblock
  \usebibmacro{title}%
  \setunit{\space}\newblock
 \printfield{howpublished}%
  \newunit
  \printlist{language}%
  \newunit\newblock
  \usebibmacro{byauthor}%
  \newunit\newblock
  \usebibmacro{bytranslator+others}%
  \newunit\newblock
  \printfield{version}%
  \newunit\newblock
  \usebibmacro{in:}%
  \usebibmacro{journal+issuetitle}%
  \newunit
  \usebibmacro{byeditor+others}%
  \newunit
  \usebibmacro{note+pages}%
  \newunit\newblock
  \iftoggle{bbx:isbn}
    {\printfield{issn}}
    {}%
  \newunit\newblock
  \usebibmacro{doi+eprint+url}%
  \newunit\newblock
  \usebibmacro{addendum+pubstate}%
  \setunit{\bibpagerefpunct}\newblock
  \usebibmacro{pageref}%
  \newunit\newblock
  \iftoggle{bbx:related}
    {\usebibmacro{related:init}%
     \usebibmacro{related}}
    {}%
  \usebibmacro{finentry}}

\DeclareBibliographyDriver{book}{%
  \usebibmacro{bibindex}%
  \usebibmacro{begentry}%
  \usebibmacro{author/editor+others/translator+others}%
  \setunit{\labelnamepunct}\newblock
  \usebibmacro{maintitle+title}%
  \setunit{\space}\newblock
 \printfield{howpublished}%
%  \newunit
%  \printlist{language}%
  \newunit\newblock
  \usebibmacro{byauthor}%
  \newunit\newblock
  \usebibmacro{byeditor+others}%
  \newunit\newblock
  \printfield{edition}%
  \newunit
  \iffieldundef{maintitle}
    {\printfield{volume}%
     \printfield{part}}
    {}%
  \newunit
  \printfield{volumes}%
  \newunit\newblock
  \usebibmacro{series+number}%
%  \newunit\newblock
%  \printfield{note}%
  \newunit\newblock
  \usebibmacro{publisher+location+date}%
  \newunit\newblock
  \usebibmacro{chapter+pages}%
  \setunit{	\addcomma\addspace}%MODIFIE
  \printfield{pagetotal}%
%  \newunit\newblock
%  \iftoggle{bbx:isbn}
%    {\printfield{isbn}}
%    {}%
 \newunit\newblock%MODIFIE
  \iftoggle{bbx:url}
    {\usebibmacro{url+urldate}}
    {}%
%  \newunit\newblock
%  \usebibmacro{doi+eprint+url}%
%  \newunit\newblock
%  \usebibmacro{addendum+pubstate}%
  \setunit{\bibpagerefpunct}\newblock
  \usebibmacro{pageref}%
%  \newunit\newblock
%  \iftoggle{bbx:related}
%    {\usebibmacro{related:init}%
%     \usebibmacro{related}}
%    {}%
  \usebibmacro{finentry}}

\DeclareBibliographyDriver{thesis}{%
  \usebibmacro{bibindex}%
  \usebibmacro{begentry}%
  \usebibmacro{author}%
  \setunit{\labelnamepunct}\newblock
  \usebibmacro{title}%
  \newunit
  \printlist{language}%
  \newunit\newblock
  \usebibmacro{byauthor}%
  \newunit\newblock
  \printfield{note}%
  \newunit\newblock
  \printfield{type}%
  \setunit{	\addcomma\addspace}%MODIFIE
  \usebibmacro{institution+location+date}%
  \newunit\newblock
  \usebibmacro{chapter+pages}%
  \setunit{	\addcomma\addspace}%MODIFIE
  \printfield{pagetotal}%
  \newunit\newblock
  \iftoggle{bbx:isbn}
    {\printfield{isbn}}
    {}%
  \newunit\newblock
  \usebibmacro{doi+eprint+url}%
  \newunit\newblock
  \usebibmacro{addendum+pubstate}%
  \setunit{\bibpagerefpunct}\newblock
  \usebibmacro{pageref}%
  \newunit\newblock
  \iftoggle{bbx:related}
    {\usebibmacro{related:init}%
     \usebibmacro{related}}
    {}%
  \usebibmacro{finentry}}

\input{macros/bibdriver_incollection.tex}
\input{macros/bibdriver_inproceedings.tex}
\DeclareBibliographyDriver{manual}{%
  \usebibmacro{bibindex}%
  \usebibmacro{begentry}%
  \usebibmacro{author/editor}%
  \setunit{\labelnamepunct}\newblock
  \usebibmacro{title}%
%   \newunit\newblock
%  \printfield{type}%
% \setunit{\addcomma\addspace}
%  \printfield{edition}%
%  \newunit
%  \printfield{version}%
  \setunit{\bibpagerefpunct}\newblock
  \usebibmacro{pageref}%
%  \newunit\newblock
%  \iftoggle{bbx:related}
%    {\usebibmacro{related:init}%
%     \usebibmacro{related}}
%    {}%
  \usebibmacro{finentry}}
\DeclareBibliographyDriver{online}{%
  \usebibmacro{bibindex}%
  \usebibmacro{begentry}%
	\usebibmacro{author}%ou printnames si suppression de la date
	\newunit\newblock
     \printfield{title}%
     \setunit{\space}\newblock
  	\printfield{howpublished}%
 \newunit\newblock%MODIFIE
  \iftoggle{bbx:url}
    {\usebibmacro{url+urldate}}
    {}%
  	\setunit{\bibpagerefpunct}\newblock
  	\usebibmacro{pageref}%
%  	\newunit\newblock
%  	\iftoggle{bbx:related}
%    {\usebibmacro{related:init}%
%     \usebibmacro{related}}
%    {}%
  	\usebibmacro{finentry}}

\DeclareBibliographyDriver{legislation}{%
  \usebibmacro{bibindex}%
  \usebibmacro{begentry}%
 %    \printnames{author}%
 %    \newunit\newblock
     \printfield{title}%
     \setunit{\space}\newblock
  	\printfield{howpublished}%
  	\iffieldundef{journaltitle}{}{%
     \newunit\newblock
     \usebibmacro{in:}%
   	\usebibmacro{journal}
  	\setunit*{\addspace}\newblock
  	\printtext{du}
  	\printdate}
     \setunit{\addcomma\space}\newblock
     \iffieldundef{volume}
       {}{
     \printfield{volume}%
     \iffieldundef{number}{}
     {\mkbibparens{\printfield{number}}}}
     \setunit{\addcomma\space}\newblock
  	 \iffieldundef{texte}
       {}{%
     \printfield{texte}}
     \setunit{\addcomma\space}\newblock
     \printfield{pages}%
  	\newunit\newblock%MODIFIE
  	\iftoggle{bbx:url}
    {\usebibmacro{url+urldate}}
    {}%
  	\setunit{\bibpagerefpunct}\newblock
  	\usebibmacro{pageref}%
%  	\newunit\newblock
%  	\iftoggle{bbx:related}
%    {\usebibmacro{related:init}%
%     \usebibmacro{related}}
%    {}%
  	\usebibmacro{finentry}}

\DeclareBibliographyDriver{newspaper}{%
  \usebibmacro{bibindex}%
  \usebibmacro{begentry}%
  \usebibmacro{author}%
  \setunit{\labelnamepunct}\newblock
  \usebibmacro{title}%
  \setunit{\space}\newblock%MODIFIE
  \printfield{howpublished}%
  \iffieldundef{journaltitle}{}{%
  \newunit\newblock
  \usebibmacro{in:}%
   \usebibmacro{journal}
   \iffieldequalstr{howpublished}{[en ligne]}{
   \setunit*{\addspace}
   \printtext{(web)}
   \setunit*{\addcomma\addspace}\newblock
   \printtext{mis en ligne le}
   \setunit*{\addspace}\newblock}{%
   \setunit*{\addspace}\newblock
   \printtext{du}
   \setunit*{\addspace}\newblock}%
   \printdate}
  \usebibmacro{volume+number+eid}%
 \setunit*{\addcomma\addspace}\newblock
 \usebibmacro{note+pages}%
%  \newunit\newblock
%  \usebibmacro{location+date}%
 \newunit\newblock%MODIFIE
  \iftoggle{bbx:url}
    {\usebibmacro{url+urldate}}
    {}%
%  \newunit\newblock
%  \usebibmacro{addendum+pubstate}%
  \setunit{\bibpagerefpunct}\newblock
  \usebibmacro{pageref}%
%  \newunit\newblock
%  \iftoggle{bbx:related}
%    {\usebibmacro{related:init}%
%     \usebibmacro{related}}
%    {}%
  \usebibmacro{finentry}}

\DeclareBibliographyDriver{database}{%
  \usebibmacro{bibindex}%
  \usebibmacro{begentry}%
  \usebibmacro{author}%
  \setunit{\labelnamepunct}\newblock
  \usebibmacro{title}%
%  \setunit{\space}\newblock
%  \printfield{howpublished}%
  \newunit\newblock
  \usebibmacro{byauthor}%
%  \newunit\newblock
%  \printfield{note}%
%  \newunit\newblock
%  \usebibmacro{location+date}%
  \newunit\newblock
  \printfield{dataedition}%
  \iffieldundef{dataedition}{\newunit\newblock}{\setunit{\addcomma\addspace}\newblock}%
  \printfield{datacover}%
%  \newunit\newblock%MODIFIE
%  \iftoggle{bbx:url}
%    {\usebibmacro{url+urldate}}
%    {}%
%  \newunit\newblock
%  \usebibmacro{addendum+pubstate}%
  \setunit{\bibpagerefpunct}\newblock
  \usebibmacro{pageref}%
%  \newunit\newblock
%  \iftoggle{bbx:related}
%    {\usebibmacro{related:init}%
%     \usebibmacro{related}}
%    {}%
  \usebibmacro{finentry}}


% rédéfinir l'affichage du volume et du numéro d'un article de revue : Journal, Volume(Issue) > bibmacro utilisé notamment pour article
\renewbibmacro*{volume+number+eid}{% Comma and parenthesis before and after journal volume
  \setunit*{\addcomma\space}% NEW
  \printfield{volume}%
  \printfield{number}%
  \printfield{eid}}

% redéfinir l'affichage d'un événement et de sa date > bibmacro utilisé notamment pour inproceedings
\renewbibmacro*{event+venue+date}{%
  \printfield{eventtitle}%
  \addcomma\addspace%MODIFIE
  \printfield{eventtitleaddon}%
  \ifboolexpr{
    test {\iffieldundef{venue}}
    and
    test {\iffieldundef{eventyear}}
  }
    {}
    {\setunit*{\addspace}%
     \printtext[]{%MODIFIE
       \printfield{venue}%
       \setunit*{\addcomma\space}%
       \printeventdate}}%
  \newunit} 
  
\renewbibmacro*{publisher+location+date}{% > bibmacro utilisé notamment pour incollection, inproceedings, book
  \printlist{location}%
  \iflistundef{publisher}
  {\setunit*{\addcomma\space}}
  {\setunit*{\addcolon\space}}%
  \printlist{publisher}%
%\setunit*{\addcomma\space}%MODIFIE
%\usebibmacro{date}%MODIFIE
\newunit}

\renewbibmacro*{institution+location+date}{% > bibmacro utilisé notamment pour report, thesis
  \printlist{institution}%
  \setunit*{\addcomma\space}%
  \printlist{location}%
  \setunit*{\addcomma\space}
  \usebibmacro{date}%
  \newunit}

% personnalisation de la commande de citation: from authoryear-icomp.cbx
\makeatletter
\renewbibmacro*{cite:labelyear+extrayear}{% Affichage de la date différente pour newspaper**
  \iffieldundef{labelyear}{}{%
  \ifentrytype{newspaper}{%
    \printtext{du} \printtext[bibhyperref]{\printdate}}{%
    \printtext[bibhyperref]{\printfield{labelyear}\printfield{extrayear}}}%
}}
% \renewbibmacro*{date}{}%
% \renewbibmacro*{issue+date}{%
%   \iffieldundef{issue}{}{\printtext[parens]{\printfield{issue}}}%
%   \newunit}
\renewbibmacro*{cite}{% pour citep (alias natbib)
  \ifentrytype{newspaper}{\printfield{journaltitle}\setunit{\addspace}\usebibmacro{cite:labelyear+extrayear}}{%**
  \ifentrytype{legislation}{\printtext[bibhyperref]{\printfield[alias]{alias}}}{%
  \iffieldundef{shorthand}
    {\ifthenelse{\ifciteibid\AND\NOT\iffirstonpage}
       {\usebibmacro{cite:ibid}}
       {\ifthenelse{\ifnameundef{labelname}\OR\iffieldundef{labelyear}}
          {\usebibmacro{cite:label}%
           \setunit{\addspace}%
           \usebibmacro{cite:labelyear+extrayear}%
           \usebibmacro{cite:reinit}}
          {\iffieldequals{namehash}{\cbx@lasthash}
             {\ifthenelse{\iffieldequals{labelyear}{\cbx@lastyear}\AND
                          \(\value{multicitecount}=0\OR\iffieldundef{postnote}\)}
                {\setunit{\addcomma}%
                 \usebibmacro{cite:extrayear}}
                {\setunit{\compcitedelim}%
                 \usebibmacro{cite:labelyear+extrayear}%
                 \savefield{labelyear}{\cbx@lastyear}}}
             {\printnames{labelname}%
              \setunit{\nameyeardelim}%
              \iffieldundef{origyear}{}{\printtext[brackets]{\printfield{origyear}}\setunit{\addspace}}%%% Ajout de la prise en compte de l'année d'origine
              \usebibmacro{cite:labelyear+extrayear}%**
              \savefield{namehash}{\cbx@lasthash}%
              \savefield{labelyear}{\cbx@lastyear}}}}}
    {\usebibmacro{cite:shorthand}%
     \usebibmacro{cite:reinit}}}}%
  \setunit{\multicitedelim}}
\makeatother
\makeatletter
\renewbibmacro*{textcite}{% pour citet (alias natbib)
  \iffieldequals{namehash}{\cbx@lasthash}
    {\iffieldundef{shorthand}
       {\ifthenelse{\iffieldequals{labelyear}{\cbx@lastyear}\AND
                    \(\value{multicitecount}=0\OR\iffieldundef{postnote}\)}
          {\setunit{\addcomma}%
           \usebibmacro{cite:extrayear}}
          {\setunit{\compcitedelim}%
           \usebibmacro{cite:labelyear+extrayear}%
           \savefield{labelyear}{\cbx@lastyear}}}
       {\setunit{\compcitedelim}%
        \usebibmacro{cite:shorthand}%
        \global\undef\cbx@lastyear}}
    {\ifnameundef{labelname}
       {\iffieldundef{shorthand}
          {\usebibmacro{cite:label}%
           \setunit{%
             \global\booltrue{cbx:parens}%
             \printdelim{nonameyeardelim}\bibopenparen}%
           \ifnumequal{\value{citecount}}{1}
             {\usebibmacro{prenote}}
             {}%
           \ifthenelse{\ifciteibid\AND\NOT\iffirstonpage}
             {\usebibmacro{cite:ibid}}
             {\usebibmacro{cite:labelyear+extrayear}}}
          {\usebibmacro{cite:shorthand}}}
       {\printnames{labelname}%
        \setunit{%
          \global\booltrue{cbx:parens}%
          \printdelim{nameyeardelim}\bibopenparen}
          \iffieldundef{origyear}{}{\printtext[brackets]{\printfield{origyear}}\setunit{\addspace}}%%% Ajout de la prise en compte de l'année d'origine% là ?
        \ifnumequal{\value{citecount}}{1}
          {\usebibmacro{prenote}}
          {}%
        \iffieldundef{shorthand}
          {\iffieldundef{labelyear}
             {\usebibmacro{cite:label}}
             {\ifthenelse{\ifciteibid\AND\NOT\iffirstonpage}
                {\usebibmacro{cite:ibid}}
                {\usebibmacro{cite:labelyear+extrayear}}}%
           \savefield{labelyear}{\cbx@lastyear}}
          {\usebibmacro{cite:shorthand}%
           \global\undef\cbx@lastyear}}%
     \stepcounter{textcitecount}%
     \savefield{namehash}{\cbx@lasthash}}%
  \setunit{%
    \ifbool{cbx:parens}
      {\bibcloseparen\global\boolfalse{cbx:parens}}
      {}%
    \textcitedelim}}
\makeatother

%Personnalisation de l'affichage des dates de publications dans les entrées
\renewbibmacro*{date+extrayear}{%
  \iffieldundef{\thefield{datelabelsource}year}
    {}
    {\printtext[parens]{%
       \iffieldundef{origyear}% this is new ...
         {}
         {\printtext[brackets]{\printorigdate}
          \setunit{\addspace}}% ... till here
       \iffieldsequal{year}{\thefield{datelabelsource}year}
         {\printlabeldateextra}%
         {\printfield{labelyear}%
          \printfield{extrayear}}}}}
  

\DeclareBibliographyDriver{report}{%
  \usebibmacro{bibindex}%
  \usebibmacro{begentry}%
  \usebibmacro{author}%
  \setunit{\labelnamepunct}\newblock
  \usebibmacro{title}%
  \setunit{\space}\newblock
 \printfield{howpublished}%
  \newunit
  \printlist{language}%
  \newunit\newblock
  \usebibmacro{byauthor}%
  \newunit\newblock
%% MODIFIE : suppression du type d'entrée
  \setunit*{\addspace}%
  \printfield{number}%
  \newunit\newblock
  \printfield{version}%
  \newunit
  \printfield{note}%
  \newunit\newblock
  \usebibmacro{institution+location+date}%
  \newunit\newblock
  \usebibmacro{chapter+pages}%
  \newunit
  \printfield{pagetotal}%
%  \newunit\newblock
%  \iftoggle{bbx:isbn}
%    {\printfield{isrn}}
%    {}%
  	\newunit\newblock%MODIFIE
  	\iftoggle{bbx:url}
    {\usebibmacro{url+urldate}}
    {}%
%  \newunit\newblock
%  \usebibmacro{addendum+pubstate}%
  \setunit{\bibpagerefpunct}\newblock
  \usebibmacro{pageref}%
%  \newunit\newblock
%  \iftoggle{bbx:related}
%    {\usebibmacro{related:init}%
%     \usebibmacro{related}}
%    {}%
  \usebibmacro{finentry}}
 % Personnalisation du style bibliographique authoryear-comp