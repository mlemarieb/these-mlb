\pdfminorversion=4
\pdfobjcompresslevel=7
\pdfcompresslevel=9

%%%%%%%%%%%%%%%%
% Encodages, langues et fonts

\usepackage[utf8]{inputenc} 
\usepackage[T1]{fontenc} 
\usepackage[english,french]{babel} % dans cet ordre, le français est défini comme la langue principale du document (documentation babel)
% Guillemets
\usepackage[
autostyle=true,
french=quotes,
maxlevel=3]{csquotes}
\MakeOuterQuote{"} % citation externe
\frenchspacing
\usepackage{lmodern}% Font
\usepackage{microtype} % typographie, recommandé avec pdflatex [stretch=10,shrink=10]
% \usepackage[np]{numprint}
\newcommand\fontcorrigee[1]{{\fontfamily{cmr}\selectfont #1}}

%%%%%%%%%%%%%%%%
% Sweave et programmation
\usepackage[noae]{Sweave}% le package ae doit être désactivé si je veux afficher les guillemets
\usepackage{filecontents}
\usepackage{xifthen}
\usepackage{xstring}
\usepackage{etoolbox}
% \usepackage{chngcntr}
\usepackage{newfloat} % créer de nouveaux environnements comme des encadrés (voir partie tikz)
\usepackage{suffix} % commande WithSuffix
\usepackage{l3regex} % expressions régulières
\usepackage{xparse}

%%%%%%%%%%%%%%%%
% Mise en forme

% Date
\usepackage[ddmmyyyy]{datetime} % Format de la date

% Symboles supplémentaires
\usepackage{marvosym} % pictogrammes tel que l'éclair utilisé dans tikz (à charger avant eurosym)
\usepackage{eurosym} % symboles monétaires
\usepackage[super]{nth} % exposants anglais
\usepackage{hologo} % logos TeX (nécessaire pour pouvoir mobiliser ces logoso dans des def)
\usepackage{pifont} % ornements
\usepackage{phaistos} % symboles anciens

% Mise en forme du texte
% \usepackage{fancyvrb} % Verbatim
\newcommand{\latin}[1]{\textit{#1}} % Expressions latines
\newcommand{\etran}[1]{\textit{#1}} % Expressions étrangères

% Better support for ragged left and right. Provides the commands \RaggedRight and \RaggedLeft. 
% Standard LaTeX commands are \raggedright and \raggedleft
% http://www.ctan.org/pkg/ragged2e
\usepackage{ragged2e}