
% Ajustements
\makeatletter
\renewcommand*\l@paragraph{\bprot@dottedtocline{4}{12em}{0em}}% Paragraph indentation
\makeatother

\makeatletter
\renewcommand*\l@figure{\@dottedtocline{1}{0em}{2.3em}}
\let\l@table\l@figure
\makeatother

\makeatletter
\@ifclassloaded{scrbook}%
  {\addtokomafont{chapterentrypagenumber}{\normalfont}}% Numéros de page non gras
  {}%
\makeatother%

%%%%%%%%%%%%%%%%%%%%%%% 
% Sommaire

\usepackage[tight]{shorttoc}% permet de créer une table des matières (détail)  et un sommaire (synthétique), option tight resserre les lignes du sommaire

%%%%%%%%%%%%%%%%%%%%%%% 
% Profondeur de numérotation

% Level for numbered captions
\setcounter{secnumdepth}{3}

% Level of chapters that appear in Table of Contents
\setcounter{tocdepth}{3} % bis wohin ins Inhaltsverzeichnis aufnehmen
% -2 no caption at all
% -1 part
% 0  chapter
% 1  section    
% 2  subsection 
% 3  subsubsection
% 4  paragraph
% 5  subparagraph

%%%%%%%%%%%%%%%%%%%%%%% 
% Liste des illustrations

% Niveau des titres des tables de figures, de tableaux (comme sections et non comme chapitres)
\makeatletter
  \renewcommand\listoffigures{%
    \section*{\listfigurename}%
    \@mkboth{Table des illustrations}%listfigurename
    {Table des illustrations}%listfigurename
    \@starttoc{lof}%
  }
  \renewcommand\listoftables{%
    \section*{\listtablename}%
    \@mkboth{Table des illustrations}%listtablename
    {Table des illustrations}%listtablename
    \@starttoc{lot}%
  } % Premier numéro de l'équation% true code
\makeatother