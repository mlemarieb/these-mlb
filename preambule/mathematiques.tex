% Mathématiques et statistiques

\newcommand{\na}{{\footnotesize NA}}% Valeurs manquantes
\newcommand{\nodata}{\textcolor{gray}{--}}% Pas de données
\newcommand{\secret}{s}% secret statistique

\usepackage{siunitx} % unités de mesure
\sisetup{%
  locale = FR, % conventions françaises
  zero-decimal-to-integer = true,% Si valeur nulle, alors écrire 0 seulement
  text-sf=\sffamily\sansmath,
  detect-family=true % nécessaire pour identifier les changements de font
}

\DeclareSIUnit{\ph}{pH}
\DeclareSIUnit{\euros}{\text{\euro}}

%*% Restreindre la notation scientifique aux très grands et très petits nombres
% source: http://tex.stackexchange.com/questions/111670/scientific-notation-only-for-large-numbers
\usepackage{expl3}% base layer of LaTeX3
\ExplSyntaxOn
    \cs_new_eq:NN \fpcmpTF \fp_compare:nTF
\ExplSyntaxOff
% Définition des seuils
\newcommand*{\ThresholdLowC}{100}
\newcommand*{\ThresholdLowU}{1}
\newcommand*{\ThresholdLowa}{0.1}
\newcommand*{\ThresholdLowb}{0.01}
\newcommand*{\ThresholdLowc}{0.001}
\newcommand*{\ThresholdLow}{0.0001}
\newcommand*{\ThresholdHigh}{1000000000}
% Reformattage de la commande num
\let\OldNum\num%
\renewcommand*{\num}[2][]{%
    \fpcmpTF{abs(#2)>\ThresholdHigh}{%
            \OldNum[scientific-notation=true,
            exponent-product = \cdot,
            round-mode = figures,
            round-precision = 3,
            #1]{#2}%
        }{%
            \fpcmpTF{abs(#2)<\ThresholdLow}{%
            \OldNum[%scientific-notation=true % au choix
            round-mode = places,
            round-precision = 3,
            round-minimum = 0.0001,#1]{#2}%
        }{%
            \fpcmpTF{abs(#2)>=\ThresholdLowC}{%
            \OldNum[%scientific-notation=true % au choix
            round-mode = places,
            round-precision = 0,
            #1]{#2}%
        }{%
            \fpcmpTF{abs(#2)>=\ThresholdLowU}{%
            \OldNum[%scientific-notation=true % au choix
            round-mode = places,
            round-precision = 1,
            #1]{#2}%
        }{%
            \fpcmpTF{abs(#2)>=\ThresholdLowb}{%
            \OldNum[scientific-notation=false,
            round-mode = places,
            round-precision = 2,
            #1]{#2}%
        }{%
            \fpcmpTF{abs(#2)>=\ThresholdLowc}{%
            \OldNum[scientific-notation=false,
            round-mode = places,
            round-precision = 3,
            #1]{#2}%
        }{%
          \OldNum[%scientific-notation=true % au choix
          round-mode = places,
          round-precision = 4,
          #1]{#2}%
        }%
        }%
        }%
        }%
        }%
    }%
}%
% Reformattage de la commande SI
\let\OldSI\SI%
\renewcommand*{\SI}[3][]{%
    \fpcmpTF{abs(#2)<=\ThresholdLow}{%
        \OldSI[scientific-notation=true,#1]{#2}{#3}%
    }{%
        \fpcmpTF{abs(#2)>=\ThresholdHigh}{%
            \OldSI[scientific-notation=true,#1]{#2}{#3}%
        }{%
            \OldSI[scientific-notation=false,#1]{#2}{#3}%
        }%
    }%
}%
% Fin

\usepackage{amsfonts} % Polices mathématiques
\usepackage{amssymb} % Symboles mathématiques
\usepackage{amsmath} % Equations et langage maths
\DeclareMathOperator{\e}{e}
\usepackage[EULERGREEK]{sansmath}%Pour obtenir le sans serif en mode math

% \setcounter{equation}{-1} % Premier numéro de l'équation

\newcommand{\var}[1]{{\MakeUppercase{#1}}}% Noms de variables: ATTENTION small caps n'accepte pas les font sans serif
\def\dispfrac{\displaystyle\frac} % redéfinition des fractions (style plus aéré)